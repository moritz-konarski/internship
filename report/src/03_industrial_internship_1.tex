\documentclass[../00_main.tex]{subfiles}

\begin{document}

\section{Industrial Internship}

\subsection{NASA Earthdata}

\subsubsection{Registration}

To access data on the NASA Earthdata platform (which includes GES DISC),
a registered account is required. The purpose of the registration is for NASA
to improve their service and to offer notifications and saved preferences
\cite{earthdata-policy}.
The Earthdata account then needs to be linked to a GES DISC account to access
the M2I3NPASM dataset I am working with. A full guide can be found under 
\href{https://www.unidata.ucar.edu/software/netcdf/}{this URL}.

\subsubsection{Downloading}

To download GES DISC Data there are many options, but \texttt{wget}
might be preferable. \texttt{wget} is a utility for downloading files from the
internet. It is generally used as a command--line tool and it is available for
most platforms. The steps necessary to set up wget to download data from GES
DISC are outlined 
\href{https://disc.gsfc.nasa.gov/data-access#mac_linux_wget}{here}. The process
simply involves setting up your login information in a local file. Then, one
needs to acquire the URLs that point to the data that is meant to be
downloaded. As described above in the section on NASA remote sensing data,
a subset of the dataset (e.g. M2I3NPASM) should be specified to significantly
reduce the download time. Then the GES DISC website will provide one or more
download links, generally in a \texttt{txt} file. This file can then be given
to \texttt{wget} (following the platform--specific instructions) to download
these files. Once the files have been downloaded, they are ready to be
analyzed. 

\subsection{Python Application Development}

This subsection is based on the notes in my internship diary and will explain
the process of developing the Python application. It will not go into great
technical detail because I want to reserve that for the description of the
finished program. Additionally, every program written during my internship
contributed to the final program and thus when I describe the final result
I will be indirectly describing the important results obtained along the way. 
The decision to work with
Python was made in our first meeting and we chose it because everyone had had
at least some experience with it and Python has a rich ecosystem of libraries
that would enable us to complete all the parts of our assignment.\newline

\subsubsection{Simple Beginnings -- 10.09. to 20.09.}

The first step was to download the netCDF data from the GES DISC website to
start working with it. I created the required account (as outlined above) and 
downloaded 3 days' worth of data of the M2I3NPASM dataset using \texttt{wget}.
Then, I developed a simple command--line python program that enabled me to more
comfortably download large quantities of files. This program used the Python
requests \cite{py-requests} library to download the files, and not \texttt{wget}. This program has
since been retired because \texttt{wget} is completely sufficient and there is
no need for this program. \newline
The next simple command--line program I wrote listed all the available variables
in a netCDF file. This was the first time I worked with these files and thus
I had to find out how it works. I chose the \texttt{netCDF4} Python library
\cite{netcdf4} for
this task because it is developed by the same group that created the file
format itself. I also had to find out how to access the variable names that are
stored in a netCDF file.\newline
Once I had familiarized myself with the file format, I wrote a simple program 
capable of creating a heat map graph of a data type that only has latitude and
longitude dimensions. This program was not flexible and most values were
hard--coded, but it forced me to explore how to plot two--dimensional heat maps
using Python. I used the \texttt{matplotlib} library \cite{py-mpl} to create the plot itself
and to save it as a picture. To create the map features I used the
\texttt{cartopy} library \cite{py-cartopy} which specializes in creating all kinds of maps.

\subsubsection{Figuring out Data Storage -- 22.09. to 01.10.}

Because M2I3NPASM data comes in 1 file per day it can be cumbersome to work
with data that covers more than a single day (many files would have to be managed
at once). Furthermore, each file includes multiple variables (because one
generally does not know which specific variable will be needed) which means
that there is unnecessary data once one decides to analyze a single variable.
To solve both of these issues, my supervisor suggested that I extract one
particular variable from multiple netCDF files and save all the data into
a single file. Now a file format for these files needed to be found.\\
The data in netCDF files are stored in multi--dimensional arrays because this
structure resembles the structure of the data most closely. Thus our data
format should also support multi--dimensional arrays. After
considering multiple alternative options (Parquet, HDF5, netCDF), with my
supervisor's advice, I decided on NPZ files, a format for the \texttt{numpy} 
Python library \cite{py-numpy}. This format is convenient because \texttt{numpy} is one of the 
most popular scientific python computing libraries and used as a backend for 
many other libraries meaning that it is widely used. Also, NPZ files natively
support multi--dimensional arrays which makes them a good fit for our data. 
The compression of NPZ files is another advantage because it saves space when 
the file becomes large. NPZ files were chosen over NPY files, which are also 
files used by the \texttt{numpy} library because NPY files are not compressed 
and thus take up more space. NPY files also only hold one single
multi--dimensional array while NPZ files can hold multiple arrays. This enables
me to store all the necessary data in a single file (the actual data plus data
for the dimensions time, latitude, longitude, level).\\
NPZ files are not self--describing and cannot hold information
about the data that they contain. To not lose the information about our data
that the netCDF files hold, another file type to store this information was
required. For this purpose, I chose JSON files \cite{json}, which can easily be read from
and written to using Python and most other programming languages. Conveniently,
the Python dictionary data type \cite{py-dict} -- a list if key--value pairs, e.g. \texttt{"age":
21} where \texttt{"age"} is the key and \texttt{21} is the value -- can be easily
converted to JSON and stored for later use. This is the approach I decided to
implement.\\
After choosing these file types, I started to develop a program that would
take every netCDF file in a directory and take the data of one variable from
each file and put it into one large multi--dimensional array. It also extracts
the values for the latitude and longitude as well as the start time of the
first file and the end time of the last file. The data, latitude, and longitude
are then saved to a single NPZ file. Then the important metadata -- the minimum
and maximum values of data and dimensions, start and end time, measurement
intervals, units, variable names -- are extracted, put into a Python
dictionary, and then saved as a JSON file.

\subsubsection{Improving the Program -- 07.10. to 22.10.2020}

After both basic versions of a heat map plotting program and a data extraction
program were developed, I worked on making them work together and
iteratively improving them. I also started to develop a graphical user
interface (GUI).

\paragraph{Data Processing.}

When the processed data was first being used in plotting some bugs became
apparent. The most severe of these was a mistake I made in handling the masked
arrays contained in the netCDF files. These arrays contain data that might not
be valid for certain values of the dimensions. In the netCDF files all of these
values are filled with a special value called the \texttt{\_FillValue}. This
value is specified in the netCDF files and can be used to process the files
correctly. If it is not handled correctly, the fill value will be interpreted
as a proper value and make any plot unusable. What I did to fix this issue was
to replace every occurrence of the fill value with the \texttt{numpy} data type
\texttt{numpy.NaN} (a specific value meaning Not a Number). This makes it simple
to ignore these values in calculations because there can be no confusion about
whether or not the numbers are valid. The fill value is also being saved to the
metadata file so that it can be used at a late date.

\paragraph{Plotting.}

The main changes to the plotting program in this part of the development were
the creation of a program that can plot time series, unifying the heat map and
time series plotting program, and enabling the plotting programs to work with
the above mentioned \texttt{numpy.NaN} values.\\
To plot a time series one has to select data for a specific point in space and
then plot all the values of that point from a start date and time to an end
date and time. The most challenging element here is the conversion of
a user--entered date and time in the format \texttt{YYYY-MM-DD H} to
a computer--readable date and time object and then to an array index that can be
used to access the data. This was achieved by reading in the user--specified
date and time as text, using a Python standard library function to convert it
into a \texttt{datetime} data type \cite{py-datetime}. Then one can find the difference between
the user--specified date and time and the start date and time of the data which
can then be used to find the index of the array that corresponds to the given
time. \\
Unifying the heat map and time series plotting involved copying the code of
both programs to a single file and then setting up the command line program to
accept the types of input required to plot each of the types of plots.\\
To get the plotting programs to work with \texttt{numpy.NaN} values, the only
change I had to make to the programs was to change the functions that find the
minimum and maximum values to functions that ignore \texttt{numpy.NaN} values.
If this is not done, any operation involving a \texttt{NaN} value will itself
result in \texttt{NaN}. Now the \texttt{NaN} values are left out of
computations and will not invalidate the results.

\paragraph{GUI.}

To create a GUI I needed to become familiar with the basics of \texttt{PyQt5},
the graphics library used in this project. PyQt5 is a Python library that
offers bindings to the C++--based Qt library. Qt covers, among others, wireless 
connectivity, web browsing, and traditional user interface (UI) development
\cite{pyqt}. 
Only the UI development part of the library will be used in this project.
A simple GUI in PyQt5 can be created in a few lines of code. The only
requirement is to create a \texttt{QApplication} \cite{pyqt-ref}, which is the main application
that is run when the code is executed. If one also wants to display something
to the user, a GUI element needs to be created. One of the simplest of such
objects
is a button, which has a text and an action that is performed when it is
clicked. The action it performs is a normal Python function and can thus do
anything a Python function could do -- display text, open another window, or
create a graph.

\subsubsection{Developing a GUI -- 22.10. to 09.11.}

Now that I had developed a working data processor program, a working
plotting program, and a very basic GUI, I could start putting the three of them
together to create a functional GUI. This process involved
re--writing my existing programs in an object--oriented programming (OOP) style. This was
necessary because as individual programs they were procedural, meaning they
would execute a set number of commands in a set order and then exit. Now 
though, a new command
could be called at any time or values might need to be modified while the
program is running, so OOP was a good idea. Additionally, PyQt5 is written
using an OOP approach, so it is just natural to use one, too. OOP also allows
code to be organized into smaller, purpose--built classes that make code
management simpler. During this process, I discovered many bugs in the original
programs that had not come up before and I was forced to rethink and improve my
existing programs. The following paragraphs give an overview of this process.

\paragraph{A better way to plot dates.} In the procedural version of the time
series plotting program, I needed to create an array of time and date pairs to
use as $x$ axis data for the time series plot. From the included metadata
I could read the number of measurements per day and through the user--specified
start and end date I had a time range. Thus I could find the number of
measurements for the time range. For each of these values for which I only knew
the index in the array, I now needed to create a piece of text that has the
date and time of the measurements. I painstakingly wrote long and hard--to--read
code that performed this task but still had some problems. Most notably, to set
the values on the axis, I manually specified them which meant that all of them
were displayed -- having one label for each day when graphing data for a whole 
year is very messy. I found that a simpler way to do this was to use a feature
of the \texttt{pandas} library \cite{pandas}. This mathematical and statistical library has
a series feature that can also create a series of dates. I simply needed to
specify the beginning and end dates and times and then the interval of time
between each measurement and a series would be generated in a single line of
code. This made my code much more readable and correct. 

\paragraph{Optimizing the data processor.} The procedural version of the data
processor reads every netCDF source file one--by--one and then copies the
desired data from it to a \texttt{numpy} array. Each of these arrays then needs
to be added to a larger array that saves all the data until it is finally saved
as an NPZ file. The simplest way to do this is to create an original array,
load the first piece of data into it, and then use the
\texttt{ndarray.append(array)}
function (\texttt{ndarray} is the \texttt{numpy} name for a multi--dimensional
array) which takes the \texttt{ndarray} and appends \texttt{array} to the end of
it. This works perfectly well, but it is very slow. While developing a GUI for
the data processor it was so slow that my operating system was giving me
warnings that my program must have crashed because it was not responding. The
reason for this dismal performance was that the \texttt{append} function does
not append in--place. That means that each time the function is called, all the
data from both the \texttt{ndarray} and the \texttt{array} are copied to a new
location. If this is done many times and especially as the main array gets
larger this operation is very slow. Thus my goal was to append the array
in--place -- meaning that I would not have to copy all the data each time -- and
I tried to find a function that would enable this. Because I could not find one
I decided to use the following approach. I know at the very beginning of the
processing how many files will be processed. I also know how many measurements
from each file I will need to copy. Thus I can create one large, empty,
appropriately sized array before the copying starts and then simply put the
data from each file in the place that it belongs. This keeps the data from
being copied each iteration and thus dramatically improved the speed of the
program.

\paragraph{Threading.} Even after optimizing the data processor and speeding it
up considerably, I was still getting warnings about my program freezing while
it was running. Additionally, when I tried to update the progress bar that
I planned to use to show the user how far the extraction had progressed, it did
not work but jumped to 100\% once the process had completed. After researching
online and reading the documentation I realized that while my data processing
was going on, the event loop that controls the program was being blocked. The
event loop is the loop that continuously checks if a button has been pressed
etc. If this loop is blocked my program becomes unresponsive, which is what
leads to warnings about it freezing and the progress bar not updating. The
solution to this problem is to use separate processes to do the time--intensive
tasks and to have the main program simply run the event loop and keep the
program responsive. PyQt5 has a feature called \texttt{QThread} which is
a process that can run on its own. It is started from the main program but then
runs in parallel with it until it is completed or otherwise stops. While it is
running, a \texttt{QThread} can send messages to the main program through
\texttt{pyqtSignal}s, which can contain any data type Python supports, from
\texttt{None} (for a simple indication that something happened) to text (maybe
a status indicator). If the data processor is started as a thread it no longer
freezes the main program and when a \texttt{pyqtSignal} is used to send
progress information to the main program the progress bar can be properly
updated. This way of implementing more time--intensive processes worked so well
that I used it for every single such case afterward.

\paragraph{Deciding on the GUI structure.} After I had created a simple GUI
that fit the needs of the data processor -- directory selection, variable
selection, button to start processing, and a progress bar -- I needed to decide
how to move forward with the GUI. I could develop small but separate GUIs for
each step in the process, create a main menu where buttons would move the user
to the desired window, or popped up a new one, but I decided to implement it
using tabs (like tabs in a web browser). Tabs would have the advantage that
I would only need a single main window and wouldn't have to switch between
multiple independent windows. This would also enable the user to look at more
than one tab, e.g. plan which data to select while some data was being
processed. To create these tabs I used the PyQt5 \texttt{QTabWidget} which
makes the creation and management of tabs easy. It allows, for example, to get
an index of which tab is currently activated and it can also activate
a specific tab to move the user to a new tab.

\paragraph{HelperFunctions for buttons and labels.} My GUI is made up of a 
relatively
small number of UI elements. The most common ones are the label (which simply
displays text) and the button (as mentioned above). There are at least two of
both on every single tab in my program. While creating them is not hard, it
does take a couple of lines of code. The basic lines needed are the
initialization, setting the desired text, and setting its size and position.
This would take about 3 lines of code (although some of these commands can be
combined in the initialization). Because I found myself writing these 3 lines
of code over and over again I created a class called
\texttt{HelperFunctions.py} that contains two functions that created a button
or a label in one line. The function itself uses the same commands that
I was using in my code, but now I didn't have to type them anymore. This step
made my code more readable and smaller in size.

\paragraph{A table to select data parameters.} After the data has been
processed from netCDF files to NPZ files, the program should allow the user to
export or plot a part or all of the data. This requires the user to be able to
specify which range of dates, latitudes, longitudes, and levels they want to
export or plot. The GUI for the selection should be precise but also simple and
intuitive. I decided to do this in the form of a table. This table allows me to
display the dimensions name, its units, and the minimum and maximum possible
values. Then there are empty fields where the user must enter the minimum and
maximum values he wants their subset of data to have. Choosing a table for this
purpose means that all the relevant data can be compactly displayed together. 

\subsubsection{Fixing Bugs -- 10.11. to 13.11.}

\paragraph{NetCDF file recognition.} When I submitted the finished program to
my supervisor, he pointed out that my program did not handle all netCDF files
correctly. The files that I had downloaded and worked with had the file
extension \texttt{.nc4} (for netCDF4) and my program only recognized these. The
files he wanted to use had the extension \texttt{.nc}, but my program rejected
them even though they were valid netCDF files. To fix this bug I had to not
only check for files with the \texttt{.nc4} extension, but also those with
the \texttt{.nc} extension. Thankfully, this bug was an easy fix.

\paragraph{Help file in the browser.} My original help section was simply
another tab that had subtabs with a paragraph of text that briefly explained
the purpose and usage of each of the components. My supervisor pointed out that
this was not very helpful nor nice to look at and he provided me with a website
template that I used to create a better--looking help section. This help
section is available in the top menu bar of the window. The website opens in
a tab in a browser and now contains pictures of the tabs of the program for
illustration.

\paragraph{Save all plot pdfs to a single file.} When my program creates
multiple files during the plotting phase, each plot is saved in a separate file.
For most file types this is the only way this works, but for the PDF format,
there are other options. My program still saved each plot to its individual
PDF. My supervisor pointed out that the better way to do this is to save the
plots to one single PDF file that contains one plot per page. I implemented
this approach and it increased reduces the number of created PDF files and
makes them easier to manage.

\paragraph{Fix the bug on Windows.} I developed my GUI on Linux and thus
I wanted to make sure it would also work on Windows. I installed all the
necessary libraries and started my program. The error I encountered came from
the fact that Windows does not allow a colon (":") to be part of a file name
while Linux does. The problem arose when files that had time values in them
were saved (a file of time "12:00" for example). On Linux, this works fine, but
on Windows, it generates an error. To fix this error, I replaced the colons
with 
dashes ("-") when the platform is Windows.

\end{document}
