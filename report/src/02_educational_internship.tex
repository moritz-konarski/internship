\documentclass[../00_main.tex]{subfiles}

\begin{document}

\section{Educational Internship}

\subsection{NASA Remote Sensing Data}

NASA Remote Sensing data is available via the Earth Science Data Systems (ESDS)
Program (see \href{https://earthdata.nasa.gov/esds}{here}). The program covers the
data acquisition, processing, and distribution with the goal to enable the
widespread use of NASA mission data. The data is available for free and their
software is publicly available as Open Source Software.\\ 
% cite https://earthdata.nasa.gov/esds
Part of ESDS is the NASA Goddard Earth Sciences (GES) Data and Information 
Services Center (DISC) which provides data on atmospheric composition, water
and energy cycles, and climate variability. 
% cite https://disc.gsfc.nasa.gov/information/documents?title=Who%20We%20Are
The GES DISC provides over 3.3 Petabytes of data,
% cite https://disc.gsfc.nasa.gov/
including the MERRA2 dataset. MERRA2 (the Modern-Era Retrospective analysis for
Research and Applications version 2) focuses on historical climate reanalysis
using satellite data. The dataset I worked with (M2I3NPASM) contains data from
January 1st, 1980 until October 1st, 2020 (at the time of writing). It covers
the whole globe with measurements taken every 3 hours.
% cite https://disc.gsfc.nasa.gov/datasets/M2I3NPASM_5.12.4/summary
M2I3NPASM includes 14 measured variables in addition to latitude, longitude,
time, and a pressure level. The measured variables include the surface pressure,
specific humidity, eastward and northward wind, and temperature. 
% cite p.12-13 https://goldsmr5.gesdisc.eosdis.nasa.gov/data/MERRA2/M2I3NPASM.5.12.4/doc/MERRA2.README.pdf
The measurements are done on a cube sphere grid (add explanation) and later
processed to fit the standard latitude and longitude grid. This processing
generally involved a bilinear interpolation of data values.\newline
% cite p.3, https://gmao.gsfc.nasa.gov/pubs/docs/Bosilovich785.pdf
The data is provided on the GES DISC website (see
\href{https://disc.gsfc.nasa.gov/datasets/M2I3NPASM_5.12.4/summary}{here}) and can be
accessed from there. The website gives the option to download only a subset of
the data by selecting a certain time range, latitude and longitude range, and
group of variables. This is advantageous because a full file for 1 day is about
1.1 GB in size.
% cite p.12 https://goldsmr5.gesdisc.eosdis.nasa.gov/data/MERRA2/M2I3NPASM.5.12.4/doc/MERRA2.README.pdf
For my internship I worked with a subset of the data that includes all
variables but is restricted to a latitude of 34\textdegree{}N to
48\textdegree{}N and a longitude of 65\textdegree{}E to 83\textdegree{}E. This
restricts the area to Kyrgyzstan and sections of all surrounding countries.
Additionally, the file size is reduced to a manageable 6 MB per file so that
a whole year of data only takes up 2.2 GB (in 365 files), the same space two
files of the complete data would take up.

\subsection{The netCDF Data Format}

The M2I3NPASM data is provided in a file format called the Network Common Data 
Form 
(netCDF). NetCDF is made up of a data format and libraries that can read and
write its data. 
% cite https://www.unidata.ucar.edu/software/netcdf/
NetCDF is developed and maintained by Unidata, a community of
research institutions, with the goal of sharing geoscience data and the tools
to use and visualize it. 
% cite https://www.unidata.ucar.edu/about/
Unidata is funded by the National Science Foundation,
a US government agency that promotes and supports science and research.
% cite https://www.nsf.gov/about/
Unidata also maintains libraries (programming interfaces) for C, Java, and
Fortran. Based on these, multiple other interfaces are available, including one
for Python.\newline
% cite https://www.unidata.ucar.edu/publications/factsheets/current/factsheet_netcdf.pdf
The netCDF data format is specifically designed to hold scientific data.
According to the Unidata website, the netCDF data format has the following
features:
% cite https://www.unidata.ucar.edu/software/netcdf/
\begin{itemize}
    \item \textbf{Self-Describing}. A netCDF file includes information about 
        the data it contains.
    \item \textbf{Portable}. A netCDF file can be accessed by computers with 
        different ways of storing integers, characters, and floating-point 
        numbers.
    \item \textbf{Scalable}. Small subsets of large datasets in various formats 
        may be accessed efficiently through netCDF interfaces, even from remote 
        servers.
    \item \textbf{Appendable}. Data may be appended to a properly structured 
        netCDF file without copying the dataset or redefining its structure.
    \item \textbf{Sharable}. One writer and multiple readers may simultaneously 
        access the same netCDF file.
    \item \textbf{Archivable}. Access to all earlier forms of netCDF data will 
        be supported by current and future versions of the software.
\end{itemize}
The NASA M2I3NPASM data is available in "classic" netCDF--4 format, meaning that
it is backwards compatible. These files have 4 dimensions:
\begin{enumerate}
    \item longitude in degrees east (meaning west is represented as negative),
    \item latitude in degrees north (making south negative),
    \item pressure in hPa,
    \item time in minutes since the first time point in a file.
\end{enumerate}
% cite p.2-3, https://gmao.gsfc.nasa.gov/pubs/docs/Bosilovich785.pdf
To be self-describing, the NASA M2I3NPASM netCDF files contain information
about, among others, the intitution that created the file, date and time of the
beginning and end of the dataset, and the minimum and maximum latitude and
longitude values. 
% cite p.5-7, https://gmao.gsfc.nasa.gov/pubs/docs/Bosilovich785.pdf
The files furthermore contain metadata for each of the
measured variables. The most important of these are the fill values
that identify missing data, the long name, a full version 
of the short variable name abbreviation, and the units of the variable.
% cite p.4-5, https://gmao.gsfc.nasa.gov/pubs/docs/Bosilovich785.pdf

\subsection{NetCDF Libraries}

Because I am working with the Python programming language for my internship
I require a netCDF library that works in that programming language. Unidata
provides an interface between the netCDF library for the C programming language
and Python. This interface is called netCDF4. It has most of the features of
the C library and enables the creation and reading of netCDF files using
Python. This interface is used in my industrial internship work to work with
the netCDF files downloaded from GES DISC.
% cite https://unidata.github.io/netcdf4-python/netCDF4/index.html

\end{document}
