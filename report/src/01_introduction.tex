\documentclass[../00_main.tex]{subfiles}

\begin{document}

\section{Introduction}

This report covers the tasks and results of my internship at the Federal State
Budgetary Institution of Science Research Station of the Russian Academy of 
Sciences in Bishkek (RS RAS). RS RAS is a subsidiary of the Ministry of Science 
and Higher Education of the Russian Federation and employs 137 people. Since 
1978 it has been researching seismic processes and developing geodynamic
models \cite{rsras-website}.\newline

% TODO: look at the paper Imashev sent and put some of that information here
The internship took place from the 7th of September 2020 to the 7th of November 
2020 and was conducted remotely due to the continuing COVID--19 pandemic. My 
AUCA supervisor for this internship was Olga Zabinyakova, Scientific Secretary 
of RS RAS and my RS RAS supervisor was Sanzhar Imashev (hereinafter called 
supervisor), Acting Head of the Laboratory for Integrated Research of 
Geodynamic Processes in Geophysical Fields.\newline

The internship was split into an educational section and an industrial section. 
According to my internship dairy form, the aim of the educational internship
was to acquire the knowledge necessary to understand the research carried out 
by RS RAS. During the industrial part of the internship, I should participate 
in a certain part of their work or in work that is similar to theirs. To 
fulfill these requirements my supervisor gave me the following tasks for my 
educational internship:
\begin{enumerate}
    \item familiarize yourself with web resources providing access to NASA 
        Earth Remote Sensing data;
    \item familiarize yourself with the scientific data format netCDF (Network
        Common Data Form);
    \item study libraries used to work with the netCDF format in various 
        computing environments.
\end{enumerate}
\noindent
For the industrial internship I was tasked to:
\begin{enumerate}
    \item register on the NASA Earthdata platform to access satellite data;
    \item develop a library for working with netCDF files in the Python
        programming language (using satellite data as an example);
    \item develop a computer application for data visualization and reanalysis 
        of NASA MERRA2 satellite data.
\end{enumerate}
These tasks are outlined in my internship diary form. In the first section of
this report, I will cover the educational part of the internship and talk about 
NASA Earth Remote Sensing data, the netCDF data format, and libraries used to 
work with netCDF files.\newline

Then, I will detail the industrial part of the internship and talk about 
registering on the NASA Earthdata platform, downloading NASA MERRA2 data, and 
the development of the computer application. The application development will 
be split into multiple sections. As part of the industrial internship section, 
the development process will be described. In a separate section, the finished 
program with all of its components will be discussed.\newline

As a practical example of using the program, I will follow the suggestion of my 
supervisor and analyze the vertical structure of the air temperature to show 
its vertical gradient, the tropopause, and the behavior around the ozone layer. 

\end{document}
