\documentclass[../00_main.tex]{subfiles}

\begin{document}

\section{Python Satellite Data Program}

This section details how to setup and use the program and how it works.
First, the setup is discussed and then each of the successive stages are
detailed. The information about different functions and ways to use Python and
the libraries are taken from the documentation. Because many different
libraries and elements work together, I will not point out every single one but
name the required one here.
% TODO: put all the libraries here

\subsection{Setup}

The programs and data required to use the program are the following:
\begin{itemize}
    \item netCDF data files (preferably from M2I3NPASM because it has been
        tested),
    \item a working installation of the Anaconda Python distribution (available
        \href{https://www.anaconda.com/products/individual}{here}).
\end{itemize}
Then one has to get the code to the program and set up the required anaconda 
environment. The code can either be downloaded from my Github repository which
can be found \href{https://github.com/moritz-konarski/internship}{here}, or I 
can provide it upon request. \\
Setting up the anaconda environment is simple.
Navigate to the \texttt{programs} folder in my internship repository. Then,
open that folder in a terminal (it must be \texttt{cmd.exe} on windows,
Powershell will not work). In the terminal, type \texttt{conda env create
--file internship\_gui.yml} (on Windows, use \texttt{conda env create
--file internship\_gui\_win.yml}). This command will create a Python
environment with all the required dependencies for my program. The installation
might take a while. The resulting environment
will have the same name as the file it was created from. To use the environment
is has to be activated with \texttt{conda activate internship\_gui} (or 
\texttt{conda activate internship\_gui\_win}. Once the environment is
activated, navigate to the \texttt{gui\_program} folder. To execute the
program, run the command \texttt{python3 main.py} (\texttt{python main.py} on
Windows). You should see a window that looks like \figref{dp01}.
\begin{figure}[H]
    \center
    \includegraphics[width=0.75\textwidth]{../graphics/dp01}
    \caption{The Data Processor Tab}
    \label{dp01}
\end{figure}

\subsection{Data Processing}

The first screen of the program is the Data Processor as seen in \figref{dp01}.
The purpose of the Data Processor is to take netCDF files as input, extract
a single variable from them and save it to a NPZ file. It is made up of two
classes, the \texttt{DataProcessor.py} class, a \texttt{QThread} that performs the
processing, and the \texttt{DataProcessorTab.py} class which handles the UI. 
The steps to using it are the following:
\begin{enumerate}
    \item Click the \textbf{Select Source Directory} button. An input dialog
        (\figref{dp02})
        will appear where you will have to select the folder where the netCDF
        files are located. The validity of the specified directory will be
        checked. 
            \begin{figure}[H]
                \center
                \includegraphics[width=0.75\textwidth]{../graphics/dp02}
                \caption{The Source Directory Selection Popup}
                \label{dp02}
            \end{figure}
    \item After \textbf{Choose} is clicked, the directory is checked to
        make sure it actually contains netCDF files. Then, the available
        variables are extracted from the first file. They are displayed in
        a drop-down list in their short form and then the selected one's
        full name is displayed (see \figref{dp03}). Here the user must 
        choose one of the variables to extract. 
        \begin{figure}[H]
            \center
            \includegraphics[width=0.75\textwidth]{../graphics/dp03}
            \caption{Data Processor with loaded source Directory}
            \label{dp04}
        \end{figure}
    \item When the choice is made, click \textbf{Extract}. A popup asking
        for a destination directory will appear where the user should
        indicate where the files should be saved. In case the extraction
        takes too long or the user changes their mind, the \textbf{Cancle
        Extraction} button will stop the extraction process. \figref{dp04}
        shows the extraction process.
        \begin{figure}[H]
            \center
            \includegraphics[width=0.75\textwidth]{../graphics/dp04}
            \caption{Extraction in Progress}
            \label{dp04}
        \end{figure}
\end{enumerate}
This process will create a new directory in the directory the user selected.
This directory will be named after the short name of the variable. It will
contain the NPZ file named \texttt{<short variable name>.npz}, which
will contain all the relevant data, and a file called
\texttt{metadata.json}, which contains all the important metadata information
that was contained in the netCDF file.\\ 
Once the extraction process is complete, the user can either continue using
this program and switch to the Data Manager tab, or take the created folder and
use the NPZ file with any \texttt{numpy}--compatible program.

\subsection{Data Management}

\begin{figure}[H]
    \center
    \includegraphics[width=0.75\textwidth]{../graphics/dm01}
    \caption{The Data Manager Tab}
    \label{dm01}
\end{figure}

The Data Manager seen in \figref{dm01} is responsible for selecting the data 
subset the user intends to use and the type of data the user intends to use.
The data subset is defined by setting minimum and maximum values for the
dimensions of the data. The type of data is either a time series, where
a location is fixed and all data for that point is available, or a heat map
where a range of latitude and longitude is selected and then all data for that
is selection is used. The Data Manager, much like the Data Processor, has two
parts, the \texttt{DataManager.py}, responsible for providing the data, and the
\texttt{DataManagerTab.py}, which is responsible for the GUI. The Data Manager
is used like this:
\begin{enumerate}
    \item Click the \textbf{Select Source Directory} button. An input dialog
        will appear where you will have to select the folder that was created
        by the Data Processor (the validity of the folder will be checked).
    \item After the directory has been successfully selected, the variable
        name, long variable name, and the unit of the variable will be
        displayed. Furthermore, the table at the bottom will be filled in with
        the minimum and maximum values that are available according to the file
        that was provided. \figref{dm02} illustrates this step.
        \begin{figure}[H]
            \center
            \includegraphics[width=0.75\textwidth]{../graphics/dm02}
            \caption{Data Manager with loaded Directory}
            \label{dm02}
        \end{figure}
    \item With the directory selected, the user now needs to choose the limits
        of the data dimensions by entering them into the table. All user
        input will be validated and in case a value is outside of the possible
        values, an error message is shown. Furthermore, all entered values will
        automatically be corrected to the closest available value. The radio
        buttons determine if time series data (see \figref{dm03}) or heat map 
        data (see \figref{dm04}) is selected and
        the table is adjusted accordingly. If the selected data does not have
        a level associated with it, the last row of the table is disabled (see
        \figref{dm05}).
        \begin{figure}[H]
            \center
            \includegraphics[width=0.75\textwidth]{../graphics/dm03}
            \caption{Data Manager in Time Series Mode}
            \label{dm03}
        \end{figure}
        \begin{figure}[H]
            \center
            \includegraphics[width=0.75\textwidth]{../graphics/dm04}
            \caption{Data Manager in Heat Map Mode}
            \label{dm04}
        \end{figure}
        \begin{figure}[H]
            \center
            \includegraphics[width=0.75\textwidth]{../graphics/dm05}
            \caption{Data Manager with data without level}
            \label{dm05}
        \end{figure}
    \item When all data fields have been filled, the user may either press the
        \textbf{Export Data} button or the \textbf{Plot Data} button. Each of
        the buttons will prepare the associated data and then switch to the
        corresponding tab.
\end{enumerate}
The values entered in the Data Manager are persistent, meaning that if the user
exports some data first and then wants to plot the same data, they only need to
go back to the Data Manager tab and press \textbf{Plot Data}.

\subsection{Data Exporting}

\begin{figure}[H]
    \center
    \includegraphics[width=0.75\textwidth]{../graphics/de01}
    \caption{The Data Export Tab}
    \label{de01}
\end{figure}

If the user pressed the \textbf{Export Data} button, they will be taken to the
Export Data tab. Again, this tabs functionality is coded in the
\texttt{DataExporter.py} class and the GUI is in the
\texttt{DataExporterTab.py} class. \figref{de01} shows what the user sees. To
use the data manager, the following actions are required:
\begin{enumerate}
    \item The user is required to choose one of the 4 options of export data
        types from the drop-down list seen in \figref{de01}. The available
        options are \texttt{.cvs} (comma separated values), \texttt{.zip}
        (compressed file), \texttt{.xlsx} (an MS Excel file), or \texttt{.html}
        (a website file to be opened in the browser). 
    \item After making their choice, the user needs to press the
        \textbf{Export} button to begin the export process. Because this
        process might produce 1000s of files depending on the data settings
        specified, the program will give the user a warning about the number of
        files that will be created (see \figref{de02}). 
        \begin{figure}[H]
            \center
            \includegraphics[width=0.5\textwidth]{../graphics/de02}
            \caption{The File Number Warning Message}
            \label{de02}
        \end{figure}
        If the user clicks \textbf{Yes}, the
        export begins. Then, the program will ask for a destination directory 
        where the files
        will be saved in a new directory called \texttt{<variable
        name>-exported}. If the process
        takes too long or the user simply wants to interrupt it, they need to
        press the \textbf{Cancel Export} button. The progress bar will show the
        extraction progress as illustrated by \figref{de03}
        \begin{figure}[H]
            \center
            \includegraphics[width=0.75\textwidth]{../graphics/de03}
            \caption{Data Export in progress}
            \label{de03}
        \end{figure}
\end{enumerate}
When heat map data is being exported, a file will be created for each time step
and for each level in each time step. This may lead to the large number of
files that the warning message (\figref{de02}) is meant for.\\
When time series data is being exported, a file will be generated for each
level but because time series have a time span, all selected time values will
be in one file. This leads to a manageable number of files.\\
The exported files will be saved in files with the following file names:
\begin{itemize}
    \item \textbf{Heat Map}: \texttt{Heat Map <long variable name> <date and
        time> <minimum latitude and longitude values>-<maximum latitude and
        longitude values> <pressure level>.<file type>} (the pressure will be
        omitted if the data does not include it). An example of this type of
        file name is \texttt{Heat Map Ertels Potential Vorticity 2019-09-01
        00:00 (34.0N, 65.0E)-(48.0N, 83.125E) 1.0 hPa}.
    \item \textbf{Time series}: \texttt{Time Series <long variable name>
        <start date and time>-<end date and time> <latitude and longitude> 
        <pressure level>.<file type>} (the pressure will be
        omitted if the data does not include it). An example of this type of
        file name is \texttt{Time Series Ertels Potential Vorticity 2019-09-01
        00:00-2019-09-10 00:00 (34.0N, 65.0E) 1.0 hPa}.
\end{itemize}
To demonstrate the result see \figref{tab:de01} which contains the exported
data for air temperature \texttt{T} in Kelvin at a pressure of 250 hPa at 
34\textdegree{}N,
65\textdegree{}E from 01.09.2019 00:00 to 30.09.2019 21:00. The data has been cut
off to save space.
\begin{figure}
\center
    \begin{tabular}{| p{5cm} | p{5cm} |} \hline
        2019-09-01 00:00:00   &  237.30182  \\\hline 
        2019-09-01 03:00:00   &  236.86859  \\\hline
        2019-09-01 06:00:00   &  236.79724  \\\hline
        2019-09-01 09:00:00   &  236.82132  \\\hline
        2019-09-01 12:00:00   &  237.73062  \\\hline
        2019-09-01 15:00:00   &  238.40472  \\\hline
        2019-09-01 18:00:00   &  238.4401   \\\hline
        2019-09-01 21:00:00   &  238.20715  \\\hline
        2019-09-02 00:00:00   &  237.49396  \\\hline
    \end{tabular}
    \caption{Temperature \texttt{T} exported as \texttt{.csv}}
    \label{tab:de01}
\end{figure}

\subsection{Data Plotting}

\begin{figure}[H]
    \center
    \includegraphics[width=0.75\textwidth]{../graphics/dpl01}
    \caption{The Data Plotter Tab for Heat Map Data}
    \label{dpl01}
\end{figure}

If the user chooses to push the \textbf{Plot Data} button in the Data Manager
tab, they will be shown the Data Plot tab. As with all other tabs, the UI is
handled by a class called \texttt{DataPlotterTab.py} and the plotting itself is
handled by the \texttt{DataPlotter.py}. The GUI the user will see if they chose
to plot heat map data is illustrated in \figref{dpl01}. If the user chose time
series data, \figref{dpl02} will be shown (the \textbf{Plot Cities} check box
will be disabled). 
\begin{figure}[H]
    \center
    \includegraphics[width=0.75\textwidth]{../graphics/dpl02}
    \caption{The Data Plotter Tab for Time Series Data}
    \label{dpl02}
\end{figure}
The steps to using this tab are very similar to the Data
Exporter tab; they are:
\begin{enumerate}
    \item The user chooses one of the 5 options of plot data
        types from the drop-down list seen in \figref{dpl01}. The available
        options are \texttt{.pdf} (portable document format), \texttt{.png}
        (portable network graphics), \texttt{.eps} (encapsulated PostScript), 
        \texttt{.svg} (scalable vector graphics), or \texttt{.jpeg}. 
    \item The next option is whether or not to use global minima and maxima or
        to use local minima and maxima. Using global minima and maxima means
        that if the user selected a range of 2 days, the minimum and maximum 
        values of all measurements in these 2 days are used as the axis limits.
        If the user chooses local minima and maxima, the minimum and maximum
        value for each individual plot will be used. The former makes is easier
        to compare multiple plots with each other, while the latter creates
        more fitting individual plots.
    \item If the user is plotting heat map data they can choose whether or not
        to plot cities on their heat map. The cities that will be plotted are
        Bishkek, Almaty, Kabul, Tashkent, and Dushanbe.
    \item After making these choices, the user needs to press the
        \textbf{Plot} button to begin the plotting process. This
        process might produce 1000s of files depending on the data settings
        specified, so the program will give the user a warning about the number 
        of files that will be created (see \figref{de02} above). 
        If the user clicks \textbf{Yes}, the
        export begins. Then, the program will ask for a destination directory 
        where the files
        will be saved in a new directory called \texttt{<variable
        name>-plotted}. If the process
        takes too long or the user simply wants to interrupt it, they need to
        press the \textbf{Cancel Plotting} button. The progress bar will show the
        extraction progress as illustrated by \figref{de03}
        \begin{figure}[H]
            \center
            \includegraphics[width=0.75\textwidth]{../graphics/dpl03}
            \caption{Data Plotter in Progress}
            \label{dpl03}
        \end{figure}
\end{enumerate}
When heat map data is being exported, a file will be created for each time step
and for each level in each time step unless the plot data type is PDF. If PDF
is chosen all the plots will be put into one large PDF file with a page for
each of the plots. If PDF is not the file type, a large number of
files may be generated.\\
When time series data is being plotted, a file will be generated for each
level but because time series have a time span, all selected time values will
be in one file. This leads to a manageable number of files.\\
The exported files will be saved in files with the save files names as
specified in the Data Export section above.\\
Sticking with the temperature example from above, the time series plot of the
same data that is shown in \figref{tab:de01} is plotted in \figref{plt:dpl01}.
\begin{figure}[h]
    \center
    \includegraphics[width=0.9\textwidth]{../graphics/plt01}
    \caption{Time Series Plot for Air temperature}
    \label{plt:dpl01}
\end{figure}
In \figref{plt:dpl02} a heat map of the air temperature on 01.09.2019 at 00:00
at a pressure of 250 hPa is graphed for a region from (34\textdegree{}N,
65\textdegree{}E) to (48\textdegree{}N, 83.125\textdegree{}E).
\begin{figure}[H]
    \center
    \includegraphics[width=0.9\textwidth]{../graphics/plt02}
    \caption{Heat Map Plot for Air temperature}
    \label{plt:dpl02}
\end{figure}
\figref{plt:dpl03} shows the same plot as \figref{plt:dpl02}, but with city
plotting enabled.
\begin{figure}[h]
    \center
    \includegraphics[width=0.9\textwidth]{../graphics/plt03}
    \caption{Heat Map Plot for Air Temperature with cities}
    \label{plt:dpl03}
\end{figure}

\subsection{Help}

The last and maybe most important section of the program is the help section.
It can be accessed by clicking on the text \textbf{Help} in the menu bar at the
top of the program window as seen in every figure of the program window, e.g.
\figref{dpl03}. This will open a local html file in a browser tab. This website
is based on a template given to me by my supervisor to create the help
section. It contains a table of contents, a short introduction to what the
program does, and then a short description of what each of the 4 tabs do,
similar to the one provided in this section. In \figref{help} the menu and
introduction are shown as they appear in a browser.
\begin{figure}[H]
    \center
    \includegraphics[width=\textwidth]{../graphics/help}
    \caption{Screen Shot the Help section}
    \label{help}
\end{figure}

\end{document}
