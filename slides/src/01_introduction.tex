\documentclass[../00_main.tex]{subfiles}

\begin{document}

\section{Introduction}

\begin{frame}{Place of Internship}
    \begin{itemize}
        \item Federal State Budgetary Institution of Science Research Station 
            of the Russian Academy of Sciences in Bishkek (RS RAS)
        \item employs 137 people
        \item founded in 1978
        \item researches seismic processes and develops geodynamic models
            \cite{rsras-website}
    \end{itemize}
\end{frame}

\begin{frame}{Internship Details}
    \begin{itemize}
        \item 7th of September 2020 to the 7th of November 2020 
        \item conducted remotely due to COVID--19
        \item AUCA supervisor was Olga Zabinyakova, Scientific Secretary 
            of RS RAS
        \item RS RAS supervisor was Sanzhar Imashev Acting Head of the 
            Laboratory for Integrated Research of Geodynamic Processes in 
            Geophysical Fields
    \end{itemize}
\end{frame}

\begin{frame}{Educational Internship Tasks}
\begin{enumerate}
    \item familiarize yourself with web resources providing access to NASA 
        Earth Remote Sensing data;
    \item familiarize yourself with the scientific data format netCDF (Network
        Common Data Form);
    \item study libraries used to work with the netCDF format in various 
        computing environments.
\end{enumerate}
\end{frame}

\begin{frame}{Industrial Internship Tasks}
\begin{enumerate}
    \item register on the NASA Earthdata platform to access satellite data;
    \item develop a library for working with netCDF files in the Python
        programming language (using satellite data as an example);
    \item develop a computer application for data visualization and reanalysis 
        of NASA MERRA2 satellite data.
\end{enumerate}
\end{frame}

\end{document}

\end{document}
